\question 设有 $n$ 个人围绕一张圆桌就座,允许旋转圆桌。用 Burnside 引理求方案数。

\textbf{解:} 

设集合 $X$ 为 $n$ 个不同的人在固定线性座位上的全排列,则 $|X| = n!$。
设作用在 $X$ 上的置换群 $G$ 为圆桌的旋转群 $C_n = \{g_0, g_1, \dots, g_{n-1}\}$,其中 $g_k$ 表示旋转 $k$ 个单位。显然 $|G| = n$。

根据 Burnside 引理,不同方案数 $N$ 为:
$$ N = \frac{1}{|G|} \sum_{g \in G} |X^g| $$

分析群元素的不动点数量 $|X^g|$:
1.  **对于恒等变换 $g_0$**:
    所有线性排列在不旋转的情况下均保持不变,故
    $$ |X^{g_0}| = n! $$
2.  **对于非平凡旋转 $g_k$ ($1 \le k \le n-1$)**:
    若一个排列在旋转 $k$ 个位置后保持不变,意味着对于任意位置 $i$,该位置上的人必须与旋转后位置 $(i+k) \pmod n$ 上的人相同。由于 $n$ 个人各不相同,一个人不可能同时占据两个不同的位置,故不存在这样的排列。因此
    $$ |X^{g_k}| = 0 $$

代入公式计算:
$$
\begin{aligned}
N &= \frac{1}{n} \left( |X^{g_0}| + \sum_{k=1}^{n-1} |X^{g_k}| \right) \\
    &= \frac{1}{n} (n! + \underbrace{0 + \dots + 0}_{n-1 \text{ 个}}) \\
    &= \frac{n!}{n} \\
    &= (n-1)!
\end{aligned}
$$

故方案数为 $(n-1)!$。

\question 设想将一个大正方形划分为四个边长为原先二分之一的小正方形,使用黑、白两种颜色对每个小正方形染色。允许的对称操作包括: (1) 大正方形在其所在平面内绕其中心旋转;(2) 将染色方案中的黑色改为白色、白色改为黑色。求不等价的染色方案数。

\textbf{解:} 

\textbf{1. 定义集合与群}
设集合 $X$ 为 $2 \times 2$ 棋盘格的染色方案,每个格子可选黑、白两色,则总数 $|X| = 2^4 = 16$。
设群 $G$ 由旋转操作 $r$(逆时针 $90^\circ$)和颜色反转操作 $c$ 生成。
群 $G$ 包含 8 个元素:
$$ G = \{e, r, r^2, r^3, c, rc, r^2c, r^3c\} $$
其中 $|G| = 8$。

\textbf{2. 分析不动点数量 $|X^g|$}
利用 Burnside 引理,我们需要计算每个群元素作用下的不动点数量:

\begin{itemize}
    \item \textbf{恒等变换 $e$}:
    所有 16 种方案均不变。
    $$ |X^e| = 16 $$

    \item \textbf{纯旋转 $r$ ($90^\circ$) 和 $r^3$ ($270^\circ$)}:
    4 个格子构成一个循环 $(1\ 2\ 3\ 4)$。若要保持不变,所有格子颜色必须相同(全黑或全白)。
    $$ |X^r| = |X^{r^3}| = 2^1 = 2 $$

    \item \textbf{纯旋转 $r^2$ ($180^\circ$)}:
    格子构成两个对角循环 $(1\ 3)(2\ 4)$。每组对角线的颜色必须相同。
    $$ |X^{r^2}| = 2^2 = 4 $$

    \item \textbf{纯颜色反转 $c$}:
    要求每个格子的颜色等于其反色(黑=白),这是不可能的。
    $$ |X^c| = 0 $$

    \item \textbf{旋转 $90^\circ$ 或 $270^\circ$ 并反色 $rc, r^3c$}:
    格子按 $1 \to 2 \to 3 \to 4 \to 1$ 变换,且每次变换颜色反转。
    这意味着 $c_2 = \bar{c}_1, c_3 = \bar{c}_2 = c_1, c_4 = \bar{c}_3 = \bar{c}_1$。
    只需确定第 1 个格子的颜色,其余随之确定(形式如黑白黑白)。
    $$ |X^{rc}| = |X^{r^3c}| = 2^1 = 2 $$

    \item \textbf{旋转 $180^\circ$ 并反色 $r^2c$}:
    格子按 $(1\ 3)(2\ 4)$ 变换并反色。
    要求 $c_3 = \bar{c}_1$ 且 $c_4 = \bar{c}_2$。
    格子 1 和 2 可自由选择颜色。
    $$ |X^{r^2c}| = 2^2 = 4 $$
\end{itemize}

\textbf{3. 计算结果}
根据 Burnside 引理,不等价的方案数 $N$ 为:
$$
\begin{aligned}
N &= \frac{1}{|G|} \sum_{g \in G} |X^g| \\
    &= \frac{1}{8} (16 + 2 + 4 + 2 + 0 + 2 + 4 + 2) \\
    &= \frac{1}{8} \times 32 \\
    &= 4
\end{aligned}
$$

故共有 4 种不等价的染色方案。

\question 给定 12 种不同的彩色,将一个经典样式的足球的每个正六边形面染成白色,每个正五边形面染成给定的彩色之一,且每种颜色只染一个面;允许旋转,求不等价的染色方案数。

\textbf{解:} 

该问题等价于将 12 种不同的颜色涂在正十二面体的 12 个面上,求旋转群下的不等价方案数。

1.  \textbf{确定群的大小}:
    足球(截角二十面体)的旋转对称群同构于正十二面体的旋转群 $G$。
    正十二面体有 12 个面,每个面有 5 种旋转方位,故群阶 $|G| = 12 \times 5 = 60$。

2.  \textbf{应用 Burnside 引理}:
    设 $X$ 为 12 种颜色在固定面上全排列的集合,则 $|X| = 12!$。
    考查群元素 $g \in G$ 的不动点数量 $|X^g|$:
    \begin{itemize}
        \item 当 $g$ 为恒等变换 $e$ 时:所有 $12!$ 种染色方案均保持不变,即 $|X^e| = 12!$。
        \item 当 $g$ 为非恒等旋转时 ($g \neq e$):由于 12 种颜色互不相同,旋转后必然有面的颜色发生改变(不可能出现“颜色 $A$ 转到位置 2 仍是颜色 $A$”的情况,因为每种颜色只用一次),故不存在不动点,即 $|X^g| = 0$。
    \end{itemize}

3.  \textbf{计算方案数}:
    $$
    N = \frac{1}{|G|} \sum_{g \in G} |X^g| = \frac{1}{60} (12! + 0 + \dots + 0) = \frac{12!}{60}
    $$
    
    即:
    $$ N = \frac{479,001,600}{60} = 7,983,360 $$

故共有 7,983,360 种不等价的染色方案。

\question
将正六面体的各棱的中点相连,切掉八个角,得到一个新的多面体。
\\
(1) 求该多面体的面数、顶点数、棱数;\\
(2) 使用红、黄、蓝 3 种颜色对该多面体的面染色,要求 4 个面为红色,4 个面为黄色,其余面为蓝色,并且所有红色面形状均相同,所有黄色面形状也均相同。允许旋转,求不等价的染色方案数;\\
(3) 用火柴搭建该多面体,允许旋转,求不等价的方案数。

\textbf{解:}

(1) 求该多面体的面数、顶点数、棱数

该多面体为截半立方体。
\begin{itemize}
    \item \textbf{面数 $F$}:原立方体 6 个面变为 6 个正方形,8 个顶点变为 8 个正三角形,故 $F = 6 + 8 = 14$。
    \item \textbf{顶点数 $V$}:顶点位于原立方体的 12 条棱中点,故 $V = 12$。
    \item \textbf{棱数 $E$}:由欧拉公式 $F + V - E = 2 \Rightarrow 14 + 12 - E = 2$,解得 $E = 24$。
\end{itemize}

(2) 求染色方案数

设旋转群为 $G$ ($|G|=24$)。面分为正方形集合 $S$ ($|S|=6$) 和三角形集合 $T$ ($|T|=8$)。
由题意,红、黄面各有 4 个且同色面形状相同,仅有以下三种情况:
\begin{itemize}
    \item \textbf{情况 1}:红染 $S$ (4红2蓝),黄染 $T$ (4黄4蓝)。
    \item \textbf{情况 2}:红染 $T$ (4红4蓝),黄染 $S$ (4黄2蓝)(与情况 1 对称)。
    \item \textbf{情况 3}:红染 $T$ (4红),黄染 $T$ (4黄),$S$ 全蓝。
\end{itemize}

分析群元素对面的置换类型($S$ 和 $T$ 分开考虑):
\begin{enumerate}
    \item \textbf{恒等变换 $e$ (1个)}:$S$ 为 $(1)^6$,$T$ 为 $(1)^8$。
    \item \textbf{绕面中心 $\pm 90^\circ$ (6个)}:$S$ 为 $(1)^2 (4)^1$,$T$ 为 $(4)^2$。
    \item \textbf{绕面中心 $180^\circ$ (3个)}:$S$ 为 $(1)^2 (2)^2$,$T$ 为 $(2)^4$。
    \item \textbf{绕对角线 $\pm 120^\circ$ (8个)}:$S$ 为 $(3)^2$,$T$ 为 $(1)^2 (3)^2$。
    \item \textbf{绕棱中心 $180^\circ$ (6个)}:$S$ 为 $(2)^3$,$T$ 为 $(2)^4$。
\end{enumerate}

\textbf{计算情况 1 的方案数 ($N_1$)}:
\begin{itemize}
    \item $e$:$S$ 选4红($\binom{6}{4}=15$),$T$ 选4黄($\binom{8}{4}=70$) $\Rightarrow 1 \times 15 \times 70 = 1050$。
    \item $\pm 90^\circ$:$S$ 中 4 循环必染红(1种),$T$ 中选 1 个循环染黄($\binom{2}{1}=2$) $\Rightarrow 6 \times 1 \times 2 = 12$。
    \item $180^\circ$(面):$S$ 需 4 红,可选 2 个 2 循环或 1 个 2 循环加 2 个不动点($1 + \binom{2}{1}=3$);$T$ 选 2 个循环染黄($\binom{4}{2}=6$) $\Rightarrow 3 \times 3 \times 6 = 54$。
    \item $\pm 120^\circ$:$S$ 中 3 循环无法凑成 4 红 $\Rightarrow 0$。
    \item $180^\circ$(棱):$S$ 选 2 个循环染红($\binom{3}{2}=3$);$T$ 选 2 个循环染黄($\binom{4}{2}=6$) $\Rightarrow 6 \times 3 \times 6 = 108$。
\end{itemize}
$$ Sum_1 = 1050 + 12 + 54 + 0 + 108 = 1224 $$
$$ N_1 = \frac{1224}{24} = 51 $$
由对称性,\textbf{情况 2} 方案数 $N_2 = 51$。

\textbf{计算情况 3 的方案数 ($N_3$)}:
$S$ 全蓝(1种),仅计算 $T$ (4红4黄):
\begin{itemize}
    \item $e$:$\binom{8}{4}=70 \Rightarrow 1 \times 70 = 70$。
    \item $\pm 90^\circ$:选 1 循环染红($\binom{2}{1}=2$) $\Rightarrow 6 \times 2 = 12$。
    \item $180^\circ$(面):选 2 循环染红($\binom{4}{2}=6$) $\Rightarrow 3 \times 6 = 18$。
    \item $\pm 120^\circ$:不动点 1 红 1 黄,3 循环 1 红 1 黄($\binom{2}{1}\binom{2}{1}=4$) $\Rightarrow 8 \times 4 = 32$。
    \item $180^\circ$(棱):选 2 循环染红($\binom{4}{2}=6$) $\Rightarrow 6 \times 6 = 36$。
\end{itemize}
$$ Sum_3 = 70 + 12 + 18 + 32 + 36 = 168 $$
$$ N_3 = \frac{168}{24} = 7 $$

总方案数:$N = 51 + 51 + 7 = 109$。

(3) 用火柴搭建的方案数

等价于对 24 条棱进行 2 染色(定向)。
分析群元素在 24 条棱上的循环类型:
\begin{itemize}
    \item $e$ (1个):$(1)^{24}$,不动点 $2^{24}$。
    \item $\pm 90^\circ$ (面) (6个):$(4)^6$,不动点 $2^6$。
    \item $180^\circ$ (面) (3个):$(2)^{12}$,不动点 $2^{12}$。
    \item $180^\circ$ (棱) (6个):$(2)^{12}$,不动点 $2^{12}$。
    \item $\pm 120^\circ$ (角) (8个):$(3)^8$,不动点 $2^8$。
\end{itemize}
(注:$180^\circ$ 旋转共 $3+6=9$ 个,均为 12 个 2 循环)

代入 Burnside 引理公式:
$$
\begin{aligned}
N &= \frac{1}{24} \left[ 1 \cdot 2^{24} + 6 \cdot 2^6 + 9 \cdot 2^{12} + 8 \cdot 2^8 \right] \\
    &= \frac{1}{24} [ 16777216 + 384 + 36864 + 2048 ] \\
    &= \frac{16816512}{24} \\
    &= 700688
\end{aligned}
$$
答:不等价的方案数为 700,688。

\question 现有一个包含 5 个节点的无向完全图,各节点之间没有区别。使用 3 种颜色对各条边进行染色,求不等价的染色方案数。

\textbf{解:}

设图的节点数为 5,边数 $m = \binom{5}{2} = 10$,颜色数 $k=3$。
节点无区别,故群 $G$ 为 5 个节点的置换群 $S_5$,群阶 $|G| = 120$。

分析 $G$ 中各置换类型在边集上诱导的循环个数 $c(g)$:

\begin{enumerate}
    \item \textbf{类型 $1^5$ (1个)}:所有 10 条边保持不变,即 $c=10$。
    \item \textbf{类型 $2^1 1^3$ (10个)}:含 1 个对换。$c = 1 (\text{对换内}) + 3 (\text{不动点间}) + 3 (\text{组间}) = 7$。
    \item \textbf{类型 $2^2 1^1$ (15个)}:含 2 个对换。$c = 2 (\text{对换内}) + 2 (\text{组间}) + 2 (\text{与不动点}) = 6$。
    \item \textbf{类型 $3^1 1^2$ (20个)}:含 1 个三轮换。$c = 1 (\text{轮换内}) + 1 (\text{不动点间}) + 2 (\text{组间}) = 4$。
    \item \textbf{类型 $3^1 2^1$ (20个)}:$c = 1 (\text{2循环内}) + 1 (\text{3循环内}) + 1 (\text{组间 lcm}(2,3)=6) = 3$。
    \item \textbf{类型 $4^1 1^1$ (30个)}:$c = 2 (\text{4循环内}) + 1 (\text{与不动点}) = 3$。
    \item \textbf{类型 $5^1$ (24个)}:10 条边被分为 2 个长度为 5 的循环,即 $c=2$。
\end{enumerate}

根据 Burnside 引理,方案数为:
$$
\begin{aligned}
N &= \frac{1}{120} \left[ 1 \cdot 3^{10} + 10 \cdot 3^7 + 15 \cdot 3^6 + 20 \cdot 3^4 + 20 \cdot 3^3 + 30 \cdot 3^3 + 24 \cdot 3^2 \right] \\
    &= \frac{1}{120} [ 59049 + 21870 + 10935 + 1620 + 1350 + 216 ] \\
    &= \frac{95040}{120} \\
    &= 792
\end{aligned}
$$

答:不等价的染色方案数为 792。

\question 现有一个 1 × 1 × 2 的小魔方,表面有 10 个色块。该魔方的两个 1 × 1 × 1 小块间由转轴连接,可以旋转;同时整个魔方也可在空间中任意旋转。使用 2 种颜色对此魔方的每个色块染色,每种颜色分别恰好染 5 个色块,求不等价的染色方案数。

\textbf{解:}

魔方由两个 $1 \times 1 \times 1$ 的小方块(记为上下层)组成,共有 $5 \times 2 = 10$ 个面。
染色条件为 5 面黑、5 面白(总排列数 $\binom{10}{5}=252$)。
群 $G$ 包含上下层的独立旋转和整体翻转,群阶 $|G| = 4 \times 4 \times 2 = 32$。

利用 Burnside 引理 $N = \frac{1}{|G|} \sum |X^g|$ 求解。

\textbf{1. 包含整体翻转的操作 (16个)}
此类操作交换上下层,导致所有面的置换循环长度均为偶数。
由于需要染 5 个黑面(奇数),无法由偶数长度的循环凑成,故不动点数为 0。

\textbf{2. 不包含整体翻转的操作 (16个)}
仅考虑两层绕轴的独立旋转,根据旋转角度组合 $(C_1, C_2)$ 分类讨论:

\begin{enumerate}
    \item \textbf{恒等变换 $(0^\circ, 0^\circ)$ (1个)}:
    循环类型 $(1)^{10}$。
    方案数:$\binom{10}{5} = 252$。

    \item \textbf{单层旋转 $180^\circ$ $(0^\circ, 180^\circ)$ 等 (2个)}:
    循环类型 $(1)^6 (2)^2$(不动层 5 个 1 循环,旋转层 1 个 1 循环 + 2 个 2 循环)。
    方案数:$\binom{2}{0}\binom{6}{5} + \binom{2}{1}\binom{6}{3} + \binom{2}{2}\binom{6}{1} = 6 + 40 + 6 = 52$。

    \item \textbf{单层旋转 $\pm 90^\circ$ $(0^\circ, \pm 90^\circ)$ 等 (4个)}:
    循环类型 $(1)^6 (4)^1$。
    方案数:取 1 个 4 循环或不取 $\Rightarrow \binom{1}{1}\binom{6}{1} + \binom{1}{0}\binom{6}{5} = 6 + 6 = 12$。

    \item \textbf{双层均为 $180^\circ$ $(180^\circ, 180^\circ)$ (1个)}:
    循环类型 $(1)^2 (2)^4$。
    方案数:$\binom{4}{2}\binom{2}{1} = 6 \times 2 = 12$。

    \item \textbf{一层 $180^\circ$,一层 $\pm 90^\circ$ (4个)}:
    循环类型 $(1)^2 (2)^2 (4)^1$。
    方案数:含 4 循环($\binom{1}{1}\binom{2}{1}$) + 不含 4 循环($\binom{2}{2}\binom{2}{1}$) $\Rightarrow 2 + 2 = 4$。

    \item \textbf{双层均为 $\pm 90^\circ$ (4个)}:
    循环类型 $(1)^2 (4)^2$。
    方案数:$\binom{2}{1}\binom{2}{1} = 4$。
\end{enumerate}

\textbf{计算总方案数}:
$$
\begin{aligned}
N &= \frac{1}{32} \left[ 1(252) + 2(52) + 4(12) + 1(12) + 4(4) + 4(4) + 16(0) \right] \\
    &= \frac{1}{32} [ 252 + 104 + 48 + 12 + 16 + 16 ] \\
    &= \frac{448}{32} \\
    &= 14
\end{aligned}
$$

答:不等价的染色方案数为 14。