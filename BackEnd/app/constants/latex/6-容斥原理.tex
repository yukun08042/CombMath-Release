\question 求方程 $x_1 + x_2 + x_3 = 40$ 的整数解数目,满足约束条件:$(6 \le x_1 \le 15, 5 \le x_2 \le 20, 10 \le x_3 \le 25)$。

\textbf{解:} 
首先利用换元法将变量下界转化为 0。令 $y_1 = x_1 - 6, y_2 = x_2 - 5, y_3 = x_3 - 10$,则原方程转化为求 $y_1 + y_2 + y_3 = 19$ 的非负整数解,且变量满足新的上限约束:
$$ y_1 \le 9, \quad y_2 \le 15, \quad y_3 \le 15 $$

利用容斥原理求解。记 $S$ 为无上限限制的解集,$A_1, A_2, A_3$ 分别为违反 $y_1, y_2, y_3$ 上限约束的解集。 
根据隔板法公式 $\binom{n+m-1}{m-1}$(此处 $n$ 为常数,$m=3$),各项计算如下:
\begin{equation*}
    \begin{aligned}
        |S| &= \binom{19+3-1}{3-1} = \binom{21}{2} = 210 \\
        |A_1| &: \text{对应 } y_1 \ge 10 \text{ 的情况,令 } n = 19-10=9, \text{ 则 } |A_1| = \binom{9+2}{2} = 55 \\
        |A_2| &: \text{对应 } y_2 \ge 16 \text{ 的情况,令 } n = 19-16=3, \text{ 则 } |A_2| = \binom{3+2}{2} = 10 \\
        |A_3| &: \text{对应 } y_3 \ge 16 \text{ 的情况,令 } n = 19-16=3, \text{ 则 } |A_3| = \binom{3+2}{2} = 10
    \end{aligned}
\end{equation*}

检查交集情况:由于 $|A_i \cap A_j|$ 需要变量和至少为 $10+16=26$ 或 $16+16=32$,均大于方程右边的 19,故所有交集项均为 0。

综上,符合条件的解的个数为:
\begin{equation*}
    \begin{aligned}
        N &= |S| - (|A_1| + |A_2| + |A_3|) \\
            &= 210 - (55 + 10 + 10) \\
            &= 135
    \end{aligned}
\end{equation*}

因此,原方程的整数解数目为:
\begin{solution}
    $$135$$
\end{solution}


\question 在 $1, 2, \ldots, 10^6$ 这些正整数中,求十进制表示中各位数字之和为 39 的数的个数。

\textbf{解:} 
首先建立数学模型。显然 $10^6$ 的各位数字之和为 1,不满足条件,故只需考虑 $0$ 至 $999,999$ 之间的整数(可视作不足 6 位的前面补 0)。
设该整数的十进制表示为 $x_1x_2x_3x_4x_5x_6$,题目等价于求满足以下方程的整数解个数:
$$ x_1 + x_2 + x_3 + x_4 + x_5 + x_6 = 39 $$
\textbf{约束条件:} $0 \le x_i \le 9, \quad (i = 1, 2, \ldots, 6)$。

利用容斥原理求解。
记 $S$ 为无上界限制(即 $x_i \ge 0$)的所有解的集合。
记 $P_k$ 为第 $k$ 个变量违规(即 $x_k \ge 10$)的性质。
根据容斥原理,满足所有约束(即不违规)的解的个数 $N$ 为:
$$ N = \sum_{k=0}^6 (-1)^k \binom{6}{k} \cdot (\text{至少 } k \text{ 个变量违规的解数}) $$

各项计算如下(使用隔板法公式 $\binom{n+m-1}{m-1}$,其中 $m=6$):
\begin{equation*}
    \begin{aligned}
        \text{总解数 } (k=0): \quad & \binom{39+6-1}{6-1} = \binom{44}{5} = 1,086,008 \\
        \text{1 个变量违规 } (k=1): \quad & \binom{6}{1} \times \binom{(39-10)+5}{5} = 6 \times \binom{34}{5} = 1,669,536 \\
        \text{2 个变量违规 } (k=2): \quad & \binom{6}{2} \times \binom{(39-20)+5}{5} = 15 \times \binom{24}{5} = 637,560 \\
        \text{3 个变量违规 } (k=3): \quad & \binom{6}{3} \times \binom{(39-30)+5}{5} = 20 \times \binom{14}{5} = 40,040 \\
        \text{4 个及以上违规 } (k \ge 4): \quad & \text{由于 } 39 - 40 < 0,\text{不存在非负整数解,故为 } 0
    \end{aligned}
\end{equation*}

代入交错求和公式进行最终计算:
\begin{equation*}
    \begin{aligned}
        N &= \binom{44}{5} - \binom{6}{1}\binom{34}{5} + \binom{6}{2}\binom{24}{5} - \binom{6}{3}\binom{14}{5} \\
            &= 1,086,008 - 1,669,536 + 637,560 - 40,040 \\
            &= 13,992
    \end{aligned}
\end{equation*}

因此,满足条件的正整数个数为:
\begin{solution}
    $$13,992$$
\end{solution}

\question 在 $1, 2, 3, 4, 5, 6, 7$ 的全排列中,至少 5 个数不在其原本位置上,求满足要求的排列数目。

\textbf{解:} 
设 $S$ 为所有排列的集合,总数为 $7!$。
题目要求“至少 5 个数不在其原本位置上”,等价于“至多有 2 个数在原本位置上”。
即,我们需要计算恰好有 0 个、1 个、2 个不动固定点)的排列数之和。

引入错排公式:设 $D_n$ 表示 $n$ 个元素的全错排(即没有一个元素在原位)的数目,公式为:
$$ D_n = n! \sum_{k=0}^n \frac{(-1)^k}{k!} $$
其中,需要用到的错排数值为:
\begin{equation*}
    \begin{aligned}
        D_5 &= 5!( \frac{1}{2} - \frac{1}{6} + \frac{1}{24} - \frac{1}{120} ) = 60 - 20 + 5 - 1 = 44 \\
        D_6 &= 6D_5 + (-1)^6 = 6 \times 44 + 1 = 265 \\
        D_7 &= 7D_6 + (-1)^7 = 7 \times 265 - 1 = 1854
    \end{aligned}
\end{equation*}

分类讨论:

\begin{enumerate}[(1)]
    \item \textbf{恰好 0 个数在原位(全错排):}
    即 7 个数全部错位,方案数为 $D_7$:
    $$ N_0 = \binom{7}{0} \times D_7 = 1 \times 1854 = 1854 $$

    \item \textbf{恰好 1 个数在原位:}
    选 1 个固定在原位,其余 6 个全错排:
    $$ N_1 = \binom{7}{1} \times D_6 = 7 \times 265 = 1855 $$

    \item \textbf{恰好 2 个数在原位:}
    选 2 个固定在原位,其余 5 个全错排:
    $$ N_2 = \binom{7}{2} \times D_5 = \frac{7 \times 6}{2} \times 44 = 21 \times 44 = 924 $$
\end{enumerate}

综上所述,满足条件的排列总数为:
\begin{solution}
    $$N = N_0 + N_1 + N_2 = 1854 + 1855 + 924 = 4633$$
\end{solution}

\question 在 $\{1, 2, \ldots, n\}$ 的全排列中,禁止出现子串 $12, 23, 34, \ldots, (n-1)n, n1$,求满足要求的排列数目。

\textbf{解:} 
设 $S$ 为 $\{1, 2, \ldots, n\}$ 的所有全排列集合,总数为 $|S| = n!$。定义 $n$ 个违规条件,再求解不满足任何 $P_i$ 的排列数。
\begin{itemize}
    \item $P_i$:排列中出现子串 $i(i+1)$,其中 $1 \le i \le n-1$;
    \item $P_n$:排列中出现子串 $n1$。
\end{itemize}
根据容斥原理:
$$ N = \sum_{k=0}^n (-1)^k S_k $$
其中 $S_k$ 是同时满足 $k$ 个性质的排列数总和。

\textbf{1. 分析 $S_k$ 的通项:}
当我们在排列中强制 $k$ 个相邻关系(即“粘合”了 $k$ 处子串)时,只要这 $k$ 个关系不构成回路,这就相当于将 $n$ 个元素“粘”成了 $n-k$ 个连通块;
此时,剩下的排列数为 $(n-k)!$。

我们需要从 $n$ 个可能的子串关系中选出 $k$ 个。
注意,在全排列中,不可能同时满足所有 $n$ 个关系,因为线性排列只有 $n-1$ 个相邻位。

因此:
\begin{itemize}
    \item 当 $0 \le k \le n-1$ 时:任意选取 $k$ 个关系都不会构成回路。选法为 $\binom{n}{k}$。
    此时方案数为:$\binom{n}{k}(n-k)!$。
    \item 当 $k = n$ 时:这意味着必须形成一个环,这在线性排列中是不可能的。
    此时方案数为:$0$。
\end{itemize}

\textbf{2. 代入容斥公式:}
$$
\begin{aligned}
    N &= \sum_{k=0}^{n-1} (-1)^k \binom{n}{k} (n-k)! \\
        &= \sum_{k=0}^{n-1} (-1)^k \frac{n!}{k!(n-k)!} (n-k)! \\
        &= \sum_{k=0}^{n-1} (-1)^k \frac{n!}{k!} \\
        &= n! \sum_{k=0}^{n-1} \frac{(-1)^k}{k!}
\end{aligned}
$$

% \textbf{注:} 此结果与错排数 $D_n = n! \sum_{k=0}^{n} \frac{(-1)^k}{k!}$ 非常相似,仅相差最后一项 $(-1)^n$。

因此,满足要求的排列数目为:
\begin{solution}
    $$ n! \sum_{k=0}^{n-1} \frac{(-1)^k}{k!} $$
\end{solution}