\question 设数列 $\{a_n\}$ 的母函数为 $A(x) = \displaystyle{\frac{4-3x}{(1-x)(1+x-x^3)}}$,定义 $b_0 = a_0, b_1 = a_1 - a_0, ..., b_n = a_n - a_{n-1}, ...$ 求数列 $\{b_n\}$ 的母函数,化简至封闭形式。

\textbf{解:} 记 $\{b_n\}$ 的母函数为 $B(x)$,尝试利用 $A(x)$ 构造 $B(x)$:
$$ A(x) = a_0 + a_1x+a_2x……2+a_3x^3+... (1)$$
$$ x\cdot A(x) = a_0x+a_1x^2+a_2x^3+a_3x^4... (2)$$
计算 (1)-(2), 得到:
$$ (1-x)\cdot A(x) = a_0+(a_1-a_0)x+(a_2-a_1)x^2+... =B(x)$$
因此,数列 $\{b_n\}$ 的母函数即 $ (1-x)\cdot A(x) $:
\begin{solution}
    $$\displaystyle{\frac{4-3x}{1+x-x^3}}$$
\end{solution}

\question 设 $a_n = n^3 (n \geq 0)$,求数列 $\{a_n\}$ 的母函数,化简至封闭形式。

\textbf{解:} 
根据母函数定义,记 $\{a_n\}$ 的母函数为:
$$ A(x) = \displaystyle{\sum_{n=0}^\infty n^3x^n} $$
注意到 $x^n$ 在 $ \displaystyle{\frac{1}{1-x}}$ 中出现,因此尝试利用以下母函数进行构造:
$$ \displaystyle{\sum_{n=0}^\infty x^n}=1+x+x^2+...=\frac{1}{1-x} $$
首先对左右式同时求导,添加一个系数 $n$:
$$ \displaystyle{\sum_{n=0}^\infty nx^{n-1}}=1+2x+3x^2+...=\frac{1}{(1-x)^2} $$
再同时乘以 $x$,令 $x$ 的次数为 $n$:
$$ \displaystyle{\sum_{n=0}^\infty nx^n}=x+2x^2+3x^3...=\frac{x}{(1-x)^2}$$
同理,再次同时求导、同时乘以$x$,重复两次即可得到:
$$ \displaystyle{\sum_{n=0}^\infty n^3x^n}=x+8x^2+27x^3+64x^4+...=\frac{x(1+4x+x^2)}{(1-x)^4} $$
因此,$\{a_n\}$ 的母函数为:
\begin{solution}
    $$\displaystyle{\frac{x+4x^2+x^3}{(1-x)^4}}$$
\end{solution}

\question 设 $ a_n = \displaystyle{\sum _{k=1}^{n+1}k^3} (n \geq 0)$,基于习题 3.2 求数列 $\{a_n\}$ 的母函数,化简至封闭形式。

\textbf{解:}     
根据定义,$\{a_n\}$ 的母函数为:
$$A(x)=\displaystyle{\sum_{n=1}^\infty(\sum _{k=1}^{n+1}k^3)x^n}=\sum _{k=1}^{n+1}k^3\cdot \sum _{n=k-1}^{n}x^n$$
其中,第二项可以看作:
$$\displaystyle{\sum _{n=k-1}^{n}x^n}=x^{k-1}\cdot(1+x+x^2+...) =\displaystyle{x^{k-1}\cdot \frac{1}{1-x}}$$
继续化简得到:
$$A(x)=\sum _{k=1}^{n+1}k^3\cdot \frac{x^{k-1}}{1-x}=\frac{1}{x(1-x)}\sum_{k=1}^\infty k^3x^k$$
由习题 3.2 可知,$\{k^3\}$ 的母函数 $\sum_{k=0}^\infty k^3x^k$ 为 $\displaystyle{\frac{x+4x^2+x^3}{(1-x)^4}}$,因此 $A(x)$ 为:
\begin{solution}
    $$\displaystyle{\frac{1+4x+x^2}{(1-x)^5}}$$
\end{solution}

\question 设 $n$ 是正整数,有不定方程 $a + b + c + d + e = n$。求此方程的满足如下全部条件的非负整数解数目:$a$ 为偶数,$b \leq 3$,$c$ 是 4 的倍数,$d \leq 1$,$e$ 无限制。

\textbf{解:}
通过构造母函数来表示每个变量的取值范围,最终将这些母函数相乘,得到总母函数 $A(x)$,其中 $x^n$ 的系数即方程的非负整数解的个数。
\begin{itemize}
    \item $A_a(x)=1+x^2+x^4+...=\displaystyle{\frac{1}{1-x^2}}$
    \item $A_b(x)=1+x+x^2+x^3=\displaystyle{\frac{1-x^4}{1-x}}$
    \item $A_c(x)=1+x^4+x^8+...=\displaystyle{\frac{1}{1-x^4}}$
    \item $A_d(x)=1+x$
    \item $A_e(x)=1+x+x^2+x^3+...=\displaystyle{\frac{1}{1-x}}$
\end{itemize}
将以上五个母函数相乘,得到:
$$A(x)=\displaystyle{\frac{1+x}{(1-x)^2(1-x^2)}}=\frac{1}{(1-x)^3}=\sum_{n=0}^{\infty} \binom{n + 3 - 1}{3 - 1} x^n=\sum_{n=0}^\infty \binom{n + 2}{2}x^n$$
因此,满足全部条件的非负整数解数目为:
\begin{solution}
    $$\binom{n + 2}{2}$$
\end{solution}

\question 有一边长为 $n$ 的大等边三角形,其内部的若干线段将其划分为边长为 1 的小等边三角形;右图展示了 $n = 4$ 时的情形。从左下角(右图中 $A$ 点)出发,沿大等边三角形及其内部线段行走至右下角(右图中 $B$ 点),过程中仅允许向右、右上或右下走。设不同的路线数目为 $a_n$,求数列 ${a_n}$ 的母函数,化简至封闭形式。
\begin{figure}[htbp]
    \centering
    \includegraphics[width=0.25\textwidth]{3-5.png}
    \label{fig:stepwise}
\end{figure}

\textbf{解:}
边长为 $n$ 的等边三角形包含了若干边长为 $i(i\leq n)$ 的小三角形,对应的 $\{a_i\}$ 也在 $\{a_n\}$ 数列中。
按照 $B$ 点的上一步的位置,可以分为两种情况:
\begin{itemize}
    \item 来自$D$:共计 $a_{n-1}$ 条路线
    \item 来自$E$:记最后一次经过底边的位置为 $C$ 点,$AC=i\ (0\leq i\leq n-1)$
        \begin{itemize}
            \item 从 $A$ 到 $C$:即边长为 $i$ 的三角形,对应 $a_{i}$ 条路线
            \item 从 $C$ 到 $E$:先向右上方一步,再向右取边长为 $n-1-i$ 的三角形,对应 $a_{n-1-i}$ 条路线
        \end{itemize}
        % 因此,从 $A$ 到 $E$ 共计 $\displaystyle{\sum_{i=0}^{n-1}a_ia_{n-1-i}}$ 条路线
\end{itemize}
综上,从 $A$ 到 $B$ 的路线数量满足递推关系:
$$a_n=a_{n-1}+\displaystyle{\sum_{i=0}^{n-1}a_ia_{n-1-i}}\ (n\geq 1)...(1)$$

记数列 $\{a_n\}$ 的母函数为 $A(x)$。根据母函数的定义:$$A(x)=\sum_{n=0}^\infty a_nx^n...(2)$$

由$(1)$,$(2)$及二项式定理,可得:$$\displaystyle{A(x)^2=\sum_{n=0}^\infty (\sum_{i=0}^na_ia_{n-i})x^n=a_0^2+\sum_{n=1}^\infty (\sum_{i=0}^na_ia_{n-i})x^n...(3)}$$

将 $(2)$,$(3)$ 左右两边同时乘以 $x$ 可得:
$$xA(x)=\sum_{n=1}^\infty a_{n-1}x^n...(5)$$
$$\displaystyle{xA(x)^2=x+\sum_{n=2}^\infty(\sum_{k=0}^{n-1}a_ka_{n-1-k})x^n}...(6)$$
由$(5)+(6)$可得:
$$xA(x)+xA(x)^2=\sum_{n=1}^\infty a_nx^n=A(x)-a_0=A(x)-1$$
考虑到 $lim_{x\rightarrow 0}A(x)=1$,解得:
\begin{solution}
    $$\displaystyle{A(x)=\frac{1-x-\sqrt{x^2-6x+1}}{2x}}$$
\end{solution}

\end{questions}