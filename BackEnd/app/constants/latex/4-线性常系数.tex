\question 设 $G_n = F_{2n}$,其中 $F_n$ 是第 $n$ 个斐波那契数($F_n = F_{n-1} + F_{n-2}, n \geq 2, F_0 = 0, F_1 = 1$)。

\begin{enumerate}[(1)]
    \item 证明:$G_n - 3G_{n-1} + G_{n-2} = 0(n \geq 2)$;
    \item 求数列 \{Gn\} 的母函数。
\end{enumerate}

\textbf{解:} 
(1) 根据题意,
    \begin{equation*}
        \begin{aligned}
            G_n &= F_{2n} \\
                &= F_{2n-1} + F_{2n-2} \\
                &= 2F_{2n-2} + F_{2n-3} \\
                &= 2F_{2n-2} + (F_{2n-2} - F_{2n-4}) \\
                &= 3F_{2n-2} - F_{2n-4} \\
                &= 3G_{n-1} - G_{n-2}
        \end{aligned}
    \end{equation*}

(2)  根据母函数定义,$g(x)=\sum_{n=0}^\infty G_nx^n$,代入 (1) 结论可得,
    \begin{equation*}
        \begin{aligned}
            g(x)&= G_0+G_1x+\sum_{n=2}^\infty(3G_{n-1}-G_{n-2})x^n \\
                &= 0+x+3\sum_{n=2}^\infty G_{n-1}x^n-\sum_{n=2}^\infty G_{n-2}x^n \\
                &= x+3x\sum_{n=1}^\infty G_nx^n-x^2\sum_{n=0}^\infty G_nx^n \\
                &= x+3x\cdot g(x)-x^2\cdot g(x)
        \end{aligned}
    \end{equation*}
    即, $$g(x)\cdot({1-3x+x^2)=x}$$
    因此, $\{G_N\}$ 的母函数为:
    \begin{solution}
        $$g(x)=\frac{x}{1-3x+x^2}$$
    \end{solution}

\question 求解递推关系:
\[\begin{cases}
a_n - 2a_{n-1} + a_{n-2} = 4,\  (n \geq 2) & ...(1)\\
a_0 = 1, a_1 = 3 & ...(2)
\end{cases}\]

\textbf{解:}    
首先,构造更高阶的齐次递推关系,对 $(1)$ 中的 $n$ 替换为 $n-1$ ,得到 $$a_{n-1}-2a_{n-2}+a_{n-3}=4 \ \  ...(3)$$
接着进行齐次化操作,计算 $(3)-(1)$,得到 $$a_n-3a_{n-1}+3a_{n-2}-a_{n-3}=0$$
根据齐次递推关系写出特征多项式 $$C(x)=x^3-3x^2+3x-1=(x-1)^3=0$$
求得特征根为 $r=1$,因此齐次方程的通解为 $$a_n=A+Bn+Cn^2$$

最后确定通解的待定系数,代入 $n=0,1,2$,得到
\[\begin{cases}
a_0=A=1,   \\
a_1=A+B+C=3, \\
a_2=A+2B+4C
\end{cases}\]

令 (1) 中 $n=2$,则 $a_2-2a_1+a_0=4 \implies a_2=9$,代入解得
\[\begin{cases}
A=1   \\
B=0 \\
C=2
\end{cases}\]

\begin{solution}
    $$a_n=1+2n^2$$
\end{solution}

\question 求解递推关系:
\[\begin{cases}
a_n = 3a_{n-1} - 2a_{n-2} + 3 \sin \displaystyle \frac{n\pi}{2}, \ (n \geq 2) &...(1)\\
a_0 = 5, a_1 = 3 &...(2)
\end{cases}\]

\textbf{解:}
首先分离齐次项和非齐次项,得到
$$a_n - 3a_{n-1} + 2a_{n-2} = 3 \sin \displaystyle \frac{n\pi}{2}$$

\textbf{1. 求特解}

设特解为同频率的正弦和余弦的线性组合:$$a_n^{(p)} = A \cos \frac{n\pi}{2} + B \sin \frac{n\pi}{2}$$ 其中 $A$ 和 $B$ 是待定常数。

将设定的 $a_n^{(p)}$ 代入原方程,展开整理可得
$$\left(-A + 3B\right) \cos \frac{n\pi}{2} - \left(B + 3A\right) \sin \frac{n\pi}{2} = 3 \sin \frac{n\pi}{2}$$

比较左右两侧的系数,得到方程组
$$\begin{cases} -A + 3B = 0 \\ -3A - B = 3 \end{cases} \implies A = -\frac{9}{10}, B = -\frac{3}{10}$$

因此特解为 $$a_n^{(p)} = -\frac{9}{10} \cos \frac{n\pi}{2} - \frac{3}{10} \sin \frac{n\pi}{2}$$

\textbf{2. 求齐次解}

设 $a_n = r^n$,代入齐次方程 $a_n - 3a_{n-1} + 2a_{n-2} = 0$,得到特征方程 $$r^2 - 3r + 2 = (r-1)(r-2) = 0$$
解得特征根 $r_1 = 1, r_2 = 2$。由于有两个相异特征根,齐次解的形式为:$$a_n^{(h)} = C \cdot r_1^n + D \cdot r_2^n = C \cdot 1^n + D \cdot 2^n = C + D \cdot 2^n$$

\textbf{3. 求通解}

$$a_n = a_n^{(h)} + a_n^{(p)}= C + D \cdot 2^n - \frac{9}{10} \cos \frac{n\pi}{2} - \frac{3}{10} \sin \frac{n\pi}{2} \ \ ...(3)$$

\textbf{4. 求 $C, D$}

将 $n=0$ 代入 $(3)$,得
$$a_0 = C + D \cdot 2^0 - \frac{9}{10} \cos(0) - \frac{3}{10} \sin(0) = 5$$
代入 $n=1$,得
$$a_1 = C + D \cdot 2^1 - \frac{9}{10} \cos(\frac{\pi}{2}) - \frac{3}{10} \sin(\frac{\pi}{2}) = 3$$

联立得到方程
$$\begin{cases} \displaystyle C + D = \frac{59}{10} \\ \displaystyle C + 2D = \frac{33}{10} \end{cases} \implies D = -\frac{13}{5}, C = \frac{17}{2}$$

因此,递推关系的解为
\begin{solution}
$$a_n = \frac{17}{2} - \frac{13}{5} \cdot 2^n - \frac{9}{10} \cos \frac{n\pi}{2} - \frac{3}{10} \sin \frac{n\pi}{2}$$     
\end{solution}


\question 计算:$\displaystyle \sum_{k=1}^n k^4$,结果化简为关于 $n$ 的多项式。

\textbf{解:} 利用差分法降次得到齐次方程,再求齐次解的系数。

\textbf{1. 建立初始差分方程}

设 $a_n = \sum_{k=1}^{n} k^4$。根据定义,我们有$$a_n - a_{n-1} = n^4$$

\textbf{2. 降低右侧多项式次数}

通过多次差分,将等式右侧的 $n^4$ 多项式转化为常数,从而得到齐次递推关系式。

第一次差分,得到 $n^3$ 级的多项式
$$(a_n - a_{n-1}) - (a_{n-1} - a_{n-2}) = n^4 - (n-1)^4$$
$$\implies a_n - 2a_{n-1} + a_{n-2} = 4n^3 - 6n^2 + 4n - 1$$

第二次差分,得到 $n^2$ 级的多项式
$$(a_n - 2a_{n-1} + a_{n-2}) - (a_{n-1} - 2a_{n-2} + a_{n-3})$$
$$= (4n^3 - 6n^2 + 4n - 1) - [4(n-1)^3 - 6(n-1)^2 + 4(n-1) - 1]$$
$$= 12n^2 - 24n + 14$$
$$\implies a_n - 3a_{n-1} + 3a_{n-2} - a_{n-3} = 12n^2 - 24n + 14$$

第三次差分,得到 $n$ 级的多项式
$$(a_n - 3a_{n-1} + 3a_{n-2} - a_{n-3}) - (a_{n-1} - 3a_{n-2} + 3a_{n-3} - a_{n-4})$$
$$= (12n^2 - 24n + 14) - [12(n-1)^2 - 24(n-1) + 14]$$
$$= 24n - 36$$
$$\implies a_n - 4a_{n-1} + 6a_{n-2} - 4a_{n-3} + a_{n-4} = 24n - 36$$

第四次差分,得到常数项
$$(a_n - 4a_{n-1} + 6a_{n-2} - 4a_{n-3} + a_{n-4}) - (a_{n-1} - 4a_{n-2} + 6a_{n-3} - 4a_{n-4} + a_{n-5})$$
$$= (24n - 36) - [24(n-1) - 36]$$
$$= 24$$
$$\implies a_n - 5a_{n-1} + 10a_{n-2} - 10a_{n-3} + 5a_{n-4} - a_{n-5} = 24$$

\textbf{3. 求解齐次递推关系式}

再进行一次差分,得到齐次方程 $$(a_n - 5a_{n-1} + \dots - a_{n-5}) - (a_{n-1} - 5a_{n-2} + \dots - a_{n-6}) = 24 - 24 = 0$$
化简可得 $$a_n - 6a_{n-1} + 15a_{n-2} - 20a_{n-3} + 15a_{n-4} - 6a_{n-5} + a_{n-6} = 0$$

特征多项式为 $$C(x) = x^6 - 6x^5 + 15x^4 - 20x^3 + 15x^2 - 6x + 1=(x-1)^6=0$$

解得重根 $r=1$,则齐次解的形式为
$$a_n^{(h)} = A_0 r^n \binom{n}{0} + A_1 r^n \binom{n}{1} + \dots + A_{r-1} r^n \binom{n}{r-1}$$

代入 $r=1$ 和 $r=6$(重数)$$a_n = A_0 + A_1 \binom{n}{1} + A_2 \binom{n}{2} + A_3 \binom{n}{3} + A_4 \binom{n}{4} + A_5 \binom{n}{5}$$

其中 $A_0, A_1, \dots, A_5$ 为待定常数。

\textbf{4. 代入初始条件求解系数}

将 $n=0, 1, 2, 3, 4, 5$ 代入通解公式 $a_n = \sum_{i=0}^5 A_i \binom{n}{i}$,得到线性方程组
$$\begin{cases}
A_0 \cdot 1 + A_1 \cdot 0 + A_2 \cdot 0 + A_3 \cdot 0 + A_4 \cdot 0 + A_5 \cdot 0 = 0 \\
A_0 \cdot 1 + A_1 \cdot 1 + A_2 \cdot 0 + A_3 \cdot 0 + A_4 \cdot 0 + A_5 \cdot 0 = 1 \\
A_0 \cdot 1 + A_1 \cdot 2 + A_2 \cdot 1 + A_3 \cdot 0 + A_4 \cdot 0 + A_5 \cdot 0 = 17 \\
A_0 \cdot 1 + A_1 \cdot 3 + A_2 \cdot 3 + A_3 \cdot 1 + A_4 \cdot 0 + A_5 \cdot 0 = 98 \\
A_0 \cdot 1 + A_1 \cdot 4 + A_2 \cdot 6 + A_3 \cdot 4 + A_4 \cdot 1 + A_5 \cdot 0 = 354 \\
A_0 \cdot 1 + A_1 \cdot 5 + A_2 \cdot 10 + A_3 \cdot 10 + A_4 \cdot 5 + A_5 \cdot 1 = 979
\end{cases}$$

解得 $$A_0 = 0, \quad A_1 = 1, \quad A_2 = 15, \quad A_3 = 50, \quad A_4 = 60, \quad A_5 = 24$$

因此,递推关系的解为
\begin{solution}
$$a_n = 0 \binom{n}{0} + 1 \binom{n}{1} + 15 \binom{n}{2} + 50 \binom{n}{3} + 60 \binom{n}{4} + 24 \binom{n}{5}$$
\end{solution}

\question 由 $A, B, C, D$ 四个字母组成允许重复的 $n$ 位字符串,其中子串 $AB$ 至少出现一次,求满足要求的字符串数目。

\textbf{解:}
根据对 $n$ 位字符串的分类讨论,建立 $a_n$ 满足的递推关系式:$$a_n = 4 a_{n-1} + 4^{n-2} - a_{n-2}$$

整理得到线性常系数非齐次递推关系
$$a_n - 4 a_{n-1} + a_{n-2} = 4^{n-2} \quad \text{...(1)}$$

\textbf{1. 求齐次解 $a_n^{(h)}$}

考虑齐次方程 $a_n - 4 a_{n-1} + a_{n-2} = 0$。设 $a_n = r^n$,代入齐次方程,得到特征方程:$$r^2 - 4r + 1 = 0$$

解得特征根:$$r_{1, 2} = \frac{4 \pm \sqrt{16-4}}{2} = 2 \pm \sqrt{3}$$

由于有两个相异特征根,齐次解的形式为
$$a_n^{(h)} = B(2+\sqrt{3})^n + C(2-\sqrt{3})^n$$

其中 $B$ 和 $C$ 是待定常数。

\textbf{2. 求特解 $a_n^{(p)}$}

非齐次项为 $4^{n-2} = \frac{1}{16} \cdot 4^n$。

由于 $r=4$ 不是特征方程 $r^2 - 4r + 1 = 0$ 的根,故可设特解的形式为
$$a_n^{(p)} = A \cdot 4^n$$

将 $a_n^{(p)}$ 代入原方程 $(1)$
$$A \cdot 4^n - 4(A \cdot 4^{n-1}) + A \cdot 4^{n-2} = 4^{n-2}$$

将各项统一以 $4^{n-2}$ 为底数
$$A \cdot 4^2 \cdot 4^{n-2} - 4 \cdot A \cdot 4^1 \cdot 4^{n-2} + A \cdot 4^{n-2} = 4^{n-2}$$
$$(16A - 16A + A) \cdot 4^{n-2} = 1 \cdot 4^{n-2}$$
$$A \cdot 4^{n-2} = 4^{n-2}$$

比较系数可得 $A = 1$。
因此特解为$$a_n^{(p)} = 1 \cdot 4^n = 4^n$$

\textbf{3. 求通解 $a_n$}

通解是齐次解与特解之和
$$a_n = a_n^{(p)} + a_n^{(h)} = 4^n + B(2+\sqrt{3})^n + C(2-\sqrt{3})^n \quad \text{...(2)}$$

\textbf{4. 确定常数 $B, C$}

根据问题的实际意义,确定初始条件
$$a_0 = 0 \quad (\text{长度为 0,不含 } AB)$$

$$a_1 = 0 \quad (\text{串 } A, B, C, D \text{,不含 } AB)$$

将 $n=0$ 代入 $(2)$ 得到
$$a_0 = 4^0 + B(2+\sqrt{3})^0 + C(2-\sqrt{3})^0 = 0$$
$$1 + B + C = 0 \quad \text{...(3)}$$

将 $n=1$ 代入 $(2)$
$$a_1 = 4^1 + B(2+\sqrt{3})^1 + C(2-\sqrt{3})^1 = 0$$
$$4 + (2+\sqrt{3})B + (2-\sqrt{3})C = 0 \quad \text{...(4)}$$

从 $(3)$ 得 $C = -1 - B$,代入 $(4)$
$$4 + (2+\sqrt{3})B + (2-\sqrt{3})(-1-B) = 0$$
$$4 + 2B + \sqrt{3}B - (2-\sqrt{3}) - (2-\sqrt{3})B = 0$$
$$4 - 2 + (2+\sqrt{3} - (2-\sqrt{3}))B = 0$$
$$2 + 2\sqrt{3} B = 0 \implies B = -\frac{2}{2\sqrt{3}} = -\frac{1}{\sqrt{3}} = -\frac{\sqrt{3}}{3}$$

将 $B$ 代回 $C = -1 - B$
$$C = -1 - (-\frac{\sqrt{3}}{3}) = \frac{\sqrt{3}}{3} - 1 = \frac{\sqrt{3}-3}{3}$$

将 $B$ 和 $C$ 代回通解 $(2)$,整理可得
$$\begin{aligned} a_n &= 4^n + \left(-\frac{3+2\sqrt{3}}{6}\right)(2+\sqrt{3})^n + \left(\frac{2\sqrt{3}-3}{6}\right)(2-\sqrt{3})^n \\ &= 4^n - \frac{\sqrt{3}}{6} \left[\frac{3}{\sqrt{3}}+2\right] (2+\sqrt{3})^n + \frac{\sqrt{3}}{6} \left[2-\frac{3}{\sqrt{3}}\right] (2-\sqrt{3})^n \\ &= 4^n - \frac{\sqrt{3}}{6} (\sqrt{3}+2)(2+\sqrt{3})^n + \frac{\sqrt{3}}{6} (2-\sqrt{3})(2-\sqrt{3})^n \\ &= 4^n - \frac{\sqrt{3}}{6} (2+\sqrt{3})^{n+1} + \frac{\sqrt{3}}{6} (2-\sqrt{3})^{n+1} \end{aligned}$$

因此,递推关系的解为
\begin{solution}$$a_n = 4^n - \frac{\sqrt{3}}{6} (2+\sqrt{3})^{n+1} + \frac{\sqrt{3}}{6} (2-\sqrt{3})^{n+1}$$\end{solution}

\question 考虑如下汉诺塔问题的变种:有 $A, B, C$ 三根柱子,初始时 $A$ 柱上有 $n$ 个圆盘,按直径从小到大的顺序编号为 $1$ 到 $n$;最终目标是将所有偶数编号的盘套在 $B$ 柱上,所有奇数编号的盘套在 $C$ 柱上。移动圆盘时的规则和汉诺塔问题相同,求所需的最小移动次数。

\textbf{解:} 设 $a_n$ 为将 $n$ 个圆盘按给定规则(偶数盘 $\to B$,奇数盘 $\to C$)从 $A$ 柱移动到目标状态所需的最小移动次数。设 $b_n = 2^n - 1$ 为原始汉诺塔问题的最小移动次数。考虑将编号为 $n$ 的圆盘移动到目标柱子的过程,此处以 $n$ 为偶数为例,该过程可分解为以下四个主要子任务,使得最大的两个圆盘 $n$ 和 $n-1$ 归位:

(1) 移动 $1 \sim n-1$ 从 $A$ 到 $C$:$a_{I} = b_{n-1}$

(2) 移动 $n$ 从 $A$ 到 $B$:$n$ 为偶数,目标为 $B$ 柱,$a_{II} = 1$

(3) 此时 $n$ 在 $B$ 上,$n-1$ 在 $C$ 上,$1 \sim n-2$ 在 $C$ 上。为了移动 $n-2$ 号圆盘到 $B$ 柱,必须将上方的  $1 \sim n-3$ 从 $C$ 移到 $A$:$$a_{III, 1} = b_{n-3}$$将 $n-2$ 从 $C$ 移到 $B$:$$a_{III, 2} = 1$$

(4) 将 $1 \sim n-3$ 移动到最终目标状态:此时 $1 \sim n-3$ 在 $A$ 柱上。根据新的规则,将 $n-3$ 个圆盘从 $A$ 移动到各自的目标柱(偶数 $\to B$,奇数 $\to C$)。$$a_{IV} = a_{n-3}$$

因此,最小移动次数 $a_n$ 满足递推关系
$$a_n = a_{I} + a_{II} + a_{III, 1} + a_{III, 2} + a_{IV} = b_{n-1} + 1 + b_{n-3} + 1 + a_{n-3} \quad (n \ge 3)$$

代入原始汉诺塔公式 $b_k = 2^k - 1$ 得到
$$a_n = (2^{n-1} - 1) + 1 + (2^{n-3} - 1) + 1 + a_{n-3} = a_{n-3} + 2^{n-1} + 2^{n-3} = 5 \cdot 2^{n-3}$$
最终得到递推关系式
$$a_n - a_{n-3} = 5 \cdot 2^{n-3}$$

\textbf{1. 求齐次解 $a_n^{(h)}$}

考虑齐次方程 $a_n - a_{n-3} = 0$。设 $a_n = r^n$,得到特征方程:$$r^3 - 1 = 0$$解得特征根 $r_1, r_2, r_3$:$$r_1 = 1$$$$r_2 = -\frac{1}{2} + \frac{\sqrt{3}}{2}i = e^{i\frac{2\pi}{3}}$$$$r_3 = -\frac{1}{2} - \frac{\sqrt{3}}{2}i = e^{-i\frac{2\pi}{3}}$$齐次解的形式为:$$a_n^{(h)} = A_0 \cdot 1^n + A_1 \cos \left(\frac{2n\pi}{3}\right) + A_2 \sin \left(\frac{2n\pi}{3}\right)$$其中 $A_0, A_1, A_2$ 为待定常数。

\textbf{2. 求特解 $a_n^{(p)}$}

非齐次项为 $5 \cdot 2^{n-3} = \frac{5}{8} \cdot 2^n$。由于 $r=2$ 不是特征根,设特解形式为 $a_n^{(p)} = A \cdot 2^n$。将其代入递推关系 $a_n - a_{n-3} = 5 \cdot 2^{n-3}$:$$A \cdot 2^n - A \cdot 2^{n-3} = 5 \cdot 2^{n-3}$$$$A \cdot 2^3 \cdot 2^{n-3} - A \cdot 2^{n-3} = 5 \cdot 2^{n-3}$$$$(8A - A) \cdot 2^{n-3} = 5 \cdot 2^{n-3}$$$$7A = 5 \implies A = \frac{5}{7}$$
因此特解为$$a_n^{(p)} = \frac{5}{7} \cdot 2^n$$

\textbf{3. 求通解 $a_n$}

通解是齐次解与特解之和:$$a_n = A_0 + A_1 \cos \left(\frac{2n\pi}{3}\right) + A_2 \sin \left(\frac{2n\pi}{3}\right) + \frac{5}{7} \cdot 2^n \quad \text{...(2)}$$

\textbf{4. 确定常数 $A_0, A_1, A_2$}

代入初始条件 $a_0 = 0, a_1 = 1, a_2 = 2$ 到通解 $(2)$,得到线性方程组:$$\begin{cases}
n=0: & a_0 = A_0 + A_1 \cos(0) + A_2 \sin(0) + \frac{5}{7} \cdot 2^0 = 0 \\
n=1: & a_1 = A_0 + A_1 \cos(\frac{2\pi}{3}) + A_2 \sin(\frac{2\pi}{3}) + \frac{5}{7} \cdot 2^1 = 1 \\
n=2: & a_2 = A_0 + A_1 \cos(\frac{4\pi}{3}) + A_2 \sin(\frac{4\pi}{3}) + \frac{5}{7} \cdot 2^2 = 2
\end{cases}$$代入 $\cos$ 和 $\sin$ 的具体值 $\left(\cos(\frac{2\pi}{3}) = -\frac{1}{2}, \sin(\frac{2\pi}{3}) = \frac{\sqrt{3}}{2}, \cos(\frac{4\pi}{3}) = -\frac{1}{2}, \sin(\frac{4\pi}{3}) = -\frac{\sqrt{3}}{2}\right)$:

$$\begin{cases}
A_0 + A_1 + 0 + \frac{5}{7} = 0 \implies A_0 + A_1 = -\frac{5}{7} \\
A_0 - \frac{1}{2}A_1 + \frac{\sqrt{3}}{2}A_2 + \frac{10}{7} = 1 \implies A_0 - \frac{1}{2}A_1 + \frac{\sqrt{3}}{2}A_2 = 1 - \frac{10}{7} = -\frac{3}{7} \\
A_0 - \frac{1}{2}A_1 - \frac{\sqrt{3}}{2}A_2 + \frac{20}{7} = 2 \implies A_0 - \frac{1}{2}A_1 - \frac{\sqrt{3}}{2}A_2 = 2 - \frac{20}{7} = -\frac{6}{7}
\end{cases}$$

解得:$$A_0 = -\frac{2}{3}, \quad A_1 = -\frac{1}{21}, \quad A_2 = \frac{\sqrt{3}}{7}$$

\textbf{5. 最终解}

将 $A_0, A_1, A_2$ 代回通解 $(2)$,得到递推关系的解为:
\begin{solution}$$a_n = -\frac{2}{3} - \frac{1}{21} \cos \left(\frac{2n\pi}{3}\right) + \frac{\sqrt{3}}{7} \sin \left(\frac{2n\pi}{3}\right) + \frac{5}{7} \cdot 2^n$$\end{solution}

\question 使用 $k$ 种字母组成长度为 $n$ 的字符串,但不允许相同字母连续出现 3 次,求方案数。

\textbf{解:} 设满足要求的 $n$ 位字符串数目为 $a_n$。根据字符串最后两位是否相同进行分类讨论,得到递推关系:$$a_n = (k-1) a_{n-1} + (k-1) a_{n-2} \quad (n \ge 3)$$

整理得齐次递推关系
$$a_n - (k-1) a_{n-1} - (k-1) a_{n-2} = 0 \quad \text{...(1)}$$

\textbf{1. 求通解}

设 $a_n = r^n$,得到特征方程
$$r^2 - (k-1)r - (k-1) = 0$$

解得特征根 $r_1, r_2$:$$r_{1, 2} = \frac{(k-1) \pm \sqrt{(k-1)^2 - 4(1)(-(k-1))}}{2} = \frac{(k-1) \pm \sqrt{(k-1)(k+3)}}{2}$$

由于 $k>1$,特征根互不相等,通解形式为:$$a_n = A \cdot r_1^n + B \cdot r_2^n$$

其中 $A, B$ 为待定常数。

\textbf{2. 代入初始条件}

实际初始条件为 $a_1 = k$ 和 $a_2 = k^2$。为了简化计算,由递推式 $a_2 = (k-1)a_1 + (k-1)a_0$ 反推得到 $a_0 = \frac{k}{k-1}$。代入 $n=0$ 和 $n=1$ 到通解中,得到
$$\begin{cases}
A \cdot r_1^0 + B \cdot r_2^0 = a_0 \\
A \cdot r_1^1 + B \cdot r_2^1 = a_1
\end{cases} \implies \begin{cases}
A + B = \frac{k}{k-1} \\
A \cdot r_1 + B \cdot r_2 = k
\end{cases}$$

\textbf{3. 求解 $A, B$}

解上述线性方程组,得到 $A$ 和 $B$ 的表达式$$\begin{aligned} A &= \frac{k}{2\sqrt{(k-1)(k+3)}} \left( r_2 - \frac{k}{k-1} \right) \cdot \frac{1}{r_2 - r_1} \\ &= \frac{k}{2\sqrt{k-1}} \left(\frac{1}{\sqrt{k-1}} - \frac{1}{\sqrt{k+3}}\right)\end{aligned}$$$$\begin{aligned} B &= \frac{k}{2\sqrt{(k-1)(k+3)}} \left( \frac{k}{k-1} - r_1 \right) \cdot \frac{1}{r_1 - r_2} \\ &= \frac{k}{2\sqrt{k-1}} \left(\frac{1}{\sqrt{k-1}} + \frac{1}{\sqrt{k+3}}\right)\end{aligned}$$

\textbf{4. 最终解}

将 $A$ 和 $B$ 代回通解,即为递推关系的解
\begin{solution}$$a_n = \frac{k}{2\sqrt{k-1}} \left(\frac{1}{\sqrt{k-1}} - \frac{1}{\sqrt{k+3}}\right) r_1^n + \frac{k}{2\sqrt{k-1}} \left(\frac{1}{\sqrt{k-1}} + \frac{1}{\sqrt{k+3}}\right) r_2^n$$其中$$r_1 = \frac{(k-1) + \sqrt{(k-1)(k+3)}}{2}, \quad r_2 = \frac{(k-1) - \sqrt{(k-1)(k+3)}}{2}$$\end{solution}

\question 设有 $m$ 个位置,一醉汉从位置 $1$ 出发,每次移动可以到达除当前位置外的任意位置,求移动 $n$ 次后仍然回到位置 $1$ 的不同路径数目。

\textbf{解:} 设 $a_n$ 为移动 $n$ 次后仍回到位置 1 的路径数。由于每次移动都必须更换位置,因此上一次必然不在位置 1, $a_n$ 等价于移动 $n-1$ 次后不在位置 1 的路径数。可得递推关系
$$a_n = (m-1)^{n-1} - a_{n-1}$$

整理得线性常系数非齐次递推关系
$$a_n + a_{n-1} = (m-1)^{n-1} \quad \text{...(1)}$$

\textbf{1. 求齐次解 $a_n^{(h)}$}

考虑齐次方程 $a_n + a_{n-1} = 0$。设 $a_n = r^n$,得到特征方程
$$r + 1 = 0 \implies r = -1$$
齐次解的形式为
$$a_n^{(h)} = A \cdot (-1)^n$$
其中 $A$ 为待定常数。

\textbf{2. 求特解}

$a_n^{(p)}$非齐次项为 $(m-1)^{n-1}$。由于 $r=(m-1)$ 不是特征根($m-1 \neq -1$),故设特解形式为 $a_n^{(p)} = B \cdot (m-1)^{n-1}$。将其代入递推关系 $(1)$
$$B \cdot (m-1)^{n-1} + B \cdot (m-1)^{n-2} = (m-1)^{n-1}$$

两边同除以 $(m-1)^{n-2}$
$$B(m-1) + B = m-1$$$$B(m-1 + 1) = m-1$$$$B m = m-1 \implies B = \frac{m-1}{m}$$
因此特解为
$$a_n^{(p)} = \frac{m-1}{m} \cdot (m-1)^{n-1}$$

\textbf{3. 求通解} 

$a_n$通解是齐次解与特解之和
$$a_n = a_n^{(h)} + a_n^{(p)} = A \cdot (-1)^n + \frac{m-1}{m} \cdot (m-1)^{n-1} \quad \text{...(2)}$$

\textbf{4. 确定常数 $A$}

确定初始条件:移动 1 次后,必须从位置 1 跳到非 1 的位置,故 $a_1 = 0$。将 $n=1$ 代入通解 $(2)$,得到
$$a_1 = A \cdot (-1)^1 + \frac{m-1}{m} \cdot (m-1)^{1-1} = 0$$$$-A + \frac{m-1}{m} \cdot 1 = 0 \implies A = \frac{m-1}{m}$$

\textbf{5. 最终解}

将 $A$ 代回通解 $(2)$,得到递推关系的解
$$a_n = \frac{m-1}{m} \cdot (-1)^n + \frac{m-1}{m} \cdot (m-1)^{n-1}$$

提取公因式 $\frac{m-1}{m}$ 得到

\begin{solution}$$a_n = \frac{m-1}{m} \left( (-1)^n + (m-1)^{n-1} \right)$$\end{solution}

\question 使用尺寸为 $1 \times 1$ 的方砖、直角边长为 $1$ 的等腰直角三角形砖,以及斜边长为 $2$ 的等腰直角三角形砖,铺设 $1 \times n$ 的路径,求:
\begin{enumerate}[(1)]
    \item 所有可能的铺砖方案数;
    \item 每一种可能的铺砖方案中使用的砖数相加,得到的砖数的总和。
\end{enumerate}

\textbf{解:}记方砖为 $S$,小三角形为 $T1$,大三角形为 $T2$。

\begin{enumerate}
    \item 设 $a_n$ 为铺设 $1 \times n$ 路径的总方案数,$b_n$ 为铺设 $1 \times n$ 路径且最后一块为 T2 大三角形砖的方案数。    
    分别考虑最后一块只占用 $1 \times 1$ 区域 / 占用 $1 \times 2$ 区域的情况,列出递推关系
    $$\begin{cases} a_n = a_{n-1} + 2b_n \\ b_n = 2a_{n-1} + b_{n-1} \end{cases}$$
    消元可得 $$a_n - 4 a_{n-1} + a_{n-2} = 0$$
    
    其特征多项式为 $C(x) = x^2 - 4x + 1 = 0$,求解特征根 $r_{1,2}$:
    $$r_{1,2} = 2 \pm \sqrt{3}$$
    
    因此通解形式为$$a_n = A(2 + \sqrt{3})^n + B(2 - \sqrt{3})^n$$
    
    代入初始条件
    \begin{itemize}
        \item $n=1$ ($1 \times 1$ 路径): 1 块 S + 2 种 T1 组合 $\implies a_1 = 3$.
        \item $n=2$ ($1 \times 2$ 路径): $a_1 \times a_1 + 2$ 种 T2 $\implies a_2 = 3^2 + 2 = 11$.
    \end{itemize}
    
    将 $a_1=3$ 和 $a_2=11$ 代入通解,解得常数 $A, B$
    $$A = \frac{3 + \sqrt{3}}{6}, \quad B = \frac{3 - \sqrt{3}}{6}$$
    
    方案总数 $a_n$ 
    \begin{solution}
        $$a_n = \frac{3 + \sqrt{3}}{6} (2 + \sqrt{3})^n + \frac{3 - \sqrt{3}}{6} (2 - \sqrt{3})^n$$
    \end{solution}

    \item 设 $A_n$ 是所有 $a_n$ 种铺设方案中使用的瓷砖数量的总和。
    $$A_n - 4 A_{n-1} + 4 A_{n-2} - A_{n-3} = 0$$
    由于递推关系相同,且解的形式中包含 $n$ 因子,通解形式为:$$A_n = C(2 + \sqrt{3})^n + D(2 - \sqrt{3})^n + E \cdot n (2 + \sqrt{3})^n + F \cdot n (2 - \sqrt{3})^n$$

    初始条件:

    \begin{itemize}
        \item $n=0$ ($1 \times 0$ 路径): $A_0 = 0$.
        \item $n=1$ ($1 \times 1$ 路径): $(1 \text{ 块} \times 1 \text{ 种}) + (2 \text{ 块} \times 2 \text{ 种}) \implies A_1 = 1 + 4 = 5$.
    \end{itemize}

    代入初始条件 $A_0=0, A_1=5$ 及辅助条件,解得:$$C = \frac{\sqrt{3}}{18}, \quad D = -\frac{\sqrt{3}}{18}, \quad E = \frac{2 + \sqrt{3}}{3}, \quad F = \frac{2 - \sqrt{3}}{3}$$

    因此瓷砖总数和 $A_n$ 为
    \begin{solution}
            $$A_n = \left(\frac{\sqrt{3}}{18} + \frac{2 + \sqrt{3}}{3} n\right)(2 + \sqrt{3})^n + \left(-\frac{\sqrt{3}}{18} + \frac{2 - \sqrt{3}}{3} n\right)(2 - \sqrt{3})^n$$      
    \end{solution}


    
\end{enumerate}

\question 用若干单位小方格和由三个单位小方格组成的“L”形铺满一个 $2 \times n$ 的方格棋盘,记所有不同可能的铺法的数目为 $a_n$。(1)求 $a_n$;(2) 求 $a_{2025}$ 的个位数字。

\begin{enumerate}
    \item 设 $a_n$ 为铺设 $2 \times n$ 网格盘的不同铺法数目。通过观察 $2 \times n$ 网格的最右侧,铺设的最后部分分为以下几种情况:
    \begin{itemize}
        \item 最后放置两个单位小方格(占据 $2 \times 1$ 区域),方案数 $1 \cdot a_{n-1}$。
        \item 最后放置一个“L”形砖和一个单位小方格(占据 $2 \times 2$ 区域):“L”形砖有 4 种放置方向,方案数 $4 \cdot a_{n-2}$。
        \item 最后放置两个“L”形砖(占据 $2 \times 3$ 区域):有两种对称的组合方式,方案数 $2 \cdot a_{n-3}$。
    \end{itemize}
    列出递推关系 $$a_n = a_{n-1} + 4a_{n-2} + 2a_{n-3}$$
    整理得齐次递推关系 $$a_n - a_{n-1} - 4a_{n-2} - 2a_{n-3} = 0$$
    因此特征方程为 $$C(x) = x^3 - x^2 - 4x - 2 = 0$$
    通解形式 $$a_n = A(-1)^n + B(1 + \sqrt{3})^n + C(1 - \sqrt{3})^n$$

    初始条件
    \begin{itemize}
        \item $n=0$ (空盘): $a_0 = 1$.
        \item $n=1$ ($2 \times 1$): 只能用 2 块 $1 \times 1$ 方格, $a_1 = 1$.
        \item $n=2$ ($2 \times 2$): 1 种 $1 \times 1$ 组合 + 4 种 “L” 形组合, $a_2 = 5$.
    \end{itemize}
    将 $a_0=1, a_1=1, a_2=5$ 代入通解,解得常数 $$A = 1, \quad B = \frac{\sqrt{3}}{3}, \quad C = -\frac{\sqrt{3}}{3}$$

    因此,方案总数 $a_n$ 为
    \begin{solution}
        $$a_n = (-1)^n + \frac{\sqrt{3}}{3} (1 + \sqrt{3})^n - \frac{\sqrt{3}}{3} (1 - \sqrt{3})^n$$
    \end{solution}

    \item 即计算 $a_{2025} \pmod{10}$。
    模 10 序列 $b_n$:记 $b_n = a_n \pmod{10}$。
    
    由于 $a_n$ 满足线性递推关系, $b_n$ 也满足相同的模 10 递推关系:$$b_n = (b_{n-1} + 4b_{n-2} + 2b_{n-3}) \pmod{10}$$
    
    初始条件 (模 10)$$b_0 = 1, \quad b_1 = 1, \quad b_2 = 5$$
    
    由于 $b_n$ 的取值范围有限,且由前三项唯一确定,序列 $b_n$ 必定是周期序列。
    通过计算,序列 $b_n$ 的周期为 $T=24$。
    
    计算 $2025$ 模周期 $24$ 的余数:$$2025 = 24 \times 84 + 9$$
    
    因此 $$a_{2025} \equiv b_{2025} \equiv b_{24 \times 84 + 9} \equiv b_9 \pmod{10}$$
    
    观察 $b_n$ 序列,当 $n=9$ 时,$b_9 = 5$,因此

    \begin{solution}
        $$a_{2025} \pmod{10} = 5$$
    \end{solution}
    
\end{enumerate}