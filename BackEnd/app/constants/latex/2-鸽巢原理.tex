\question 有 7 个互不相同的正整数,证明其中至少存在 2 个正整数 a, b 使得 a + b 或 a − b 能被 10 除尽。

\textbf{证明:} 任意正整数的个位数都在0-9之间,因此可以构造如下7个鸽巢:
$$\{0\},\{5\},\{1,9\},\{2,8\},\{3,7\},\{4,6\}$$
将这 6 个集合看作鸽巢,7 个正整数(对应的个位数)看作鸽子,则根据鸽巢原理,至少有一个鸽巢中有两个及以上的鸽子。
\begin{itemize}
    \item 如果该鸽巢为 $\{0\}$  或 $\{5\}$,则 a+b 和 a-b 都能被 10 除尽
    \item 如果该鸽巢为 $\{1,9\},\{2,8\},\{3,7\},\{4,6\}$ 中的一个:
        \begin{itemize}
            \item 若 a,b 的个位数相同,则 a-b 可以被 10 除尽。
            \item 若 a,b 的个位数不同,则 a+b 可以被 10 除尽。
        \end{itemize}
\end{itemize}
综上,命题得证。

\question 将集合 $\{i \in Z\ |\ 1 \leq i \leq 326\}$ 划分为 5 个子集,证明必有一个子集,其中存在一个数能表示为这个子集中两个数的差。

\textbf{证明:} 
假设命题为假。
即,存在一种划分 $S = A_1 \cup A_2 \cup A_3 \cup A_4 \cup A_5$($A_i$ 互不相交),
使得对于任意子集 $A_k$ ($k=1, \dots, 5$),若 $x, y \in A_k$ 且 $x > y$,则 $(x - y) \notin A_k$。

\begin{enumerate}
    \item 根据鸽巢原理,至少有一个子集包含 $\lceil 326/5 \rceil = 66$ 个元素。
    不妨设 $|A_1| \geq 66$。设 $A_1' = \{a_1, a_2, \ldots, a_{66}\} \subseteq A_1$,且 $a_1 < a_2 < \cdots < a_{66}$。

    \item 我们递归地定义一系列差集:    
    \begin{itemize}
        \item 
        构造 $B$: 令 $B = \{a_i - a_1 \mid i=2, \dots, 66\}$。
        则 $|B| = 65$。根据假设, $a_i, a_1 \in A_1 \implies (a_i - a_1) \notin A_1$。
        所以 $B \cap A_1 = \varnothing$,即 $B \subseteq A_2 \cup A_3 \cup A_4 \cup A_5$。
        
        \item
        构造 $C$: 由鸽巢原理, $B$ 中至少有一子集包含 $\lceil 65/4 \rceil = 17$ 个元素。
        不妨设 $A_2' = \{b_1, \dots, b_{17}\} \subseteq B \cap A_2$。
        令 $C = \{b_i - b_1 \mid i=2, \dots, 17\}$。则 $|C| = 16$。
        根据假设,$C \cap A_2 = \varnothing$。
        同时, $C$ 中元素 $c = b_i - b_1 = (a_p - a_1) - (a_q - a_1) = a_p - a_q$。由于 $a_p, a_q \in A_1$,根据假设 $c \notin A_1$。
        所以 $C \subseteq A_3 \cup A_4 \cup A_5$。
        
        \item
        构造 $D$: 由鸽巢原理, $C$ 中至少有一子集包含 $\lceil 16/3 \rceil = 6$ 个元素。
        不妨设 $A_3' = \{c_1, \dots, c_6\} \subseteq C \cap A_3$。
        令 $D = \{c_i - c_1 \mid i=2, \dots, 6\}$。则 $|D| = 5$。
        根据假设,$D \cap A_3 = \varnothing$。
        $D$ 中元素 $d = c_i - c_1 = b_p - b_q = a_r - a_s$。根据假设,$d \notin A_2$ 且 $d \notin A_1$。
        所以 $D \subseteq A_4 \cup A_5$。
        
        \item
        构造 $E$: 由鸽巢原理, $D$ 中至少有一子集包含 $\lceil 5/2 \rceil = 3$ 个元素。
        不妨设 $A_4' = \{d_1, d_2, d_3\} \subseteq D \cap A_4$。
        令 $E = \{d_2 - d_1, d_3 - d_1\}$。则 $|E| = 2$。
        根据假设,$E \cap A_4 = \varnothing$。
        $E$ 中元素 $e = d_i - d_1 = c_p - c_q = b_r - b_s = a_u - a_v$。根据假设,$e \notin A_3, e \notin A_2, e \notin A_1$。
        所以 $E \subseteq A_5$。
    \end{itemize}

    \item 得到最终矛盾:
    \begin{itemize}
        \item 
        设 $E = \{e_1, e_2\} \subseteq A_5$,其中 $e_1 < e_2$。
        令 $f = e_2 - e_1$。
        根据假设,$e_1, e_2 \in A_5 \implies f \notin A_5$。
        
        \item 
        同时, $f = e_2 - e_1 = (d_3 - d_1) - (d_2 - d_1) = d_3 - d_2$。
        由于 $d_3, d_2 \in A_4'$ ( $A_4'$ 是 $A_4$ 的子集),根据假设 $f \notin A_4$。
        
        \item 
        通过类似的传递, $f$ 也可以表示为 $A_3$ 中某两元素之差、 $A_2$ 中某两元素之差、 $A_1$ 中某两元素之差。
        因此,根据假设,$f \notin A_3, f \notin A_2, f \notin A_1$。
        
        \item 
        我们得出 $f \notin A_1$ 且 $f \notin A_2$ 且 $f \notin A_3$ 且 $f \notin A_4$ 且 $f \notin A_5$。
        这意味着 $f \notin (A_1 \cup \dots \cup A_5)$,即 $f \notin S$。
        
        \item 
        但是 $f = e_2 - e_1$ 是两个 $S$ 中元素的差,且 $e_2 > e_1$,所以 $f$ 是一个正整数。
        由于 $e_2 \in E \subset D \subset C \subset B \subset S$, $e_2 \le 325$ (最大可能是 $326-1$)。
        因此 $f = e_2 - e_1 \ge 1$,且 $f < e_2 \le 325$。
        所以 $f$ 必定在 $S = \{1, \dots, 326\}$ 中。
    \end{itemize}
    我们得出了 $f \in S$ 且 $f \notin S$ 这一矛盾。
\end{enumerate}
因此,最初的假设不成立,命题得证。

\question $(m + 1)$ 行、 $\left[ m \displaystyle \binom{m+1}{2} + 1 \right]$ 列的方格,用 $m$ 种颜色给每个方格染色,证明必能找出一个由方格组成的矩形,其四角的方格颜色相同。

\begin{enumerate}
    \item 证明每一列中存在同色方格对:
    由于方格阵列有 $R = m+1$ 行,且仅使用 $m$ 种颜色进行染色。根据抽屉原理(鸽子 $m+1$ 个,鸽笼 $m$ 个),每一列中至少有一种颜色出现2次,即每一列中至少存在一个同色垂直边(同色方格对)。

    \item 确定鸽笼和鸽子数量:
    我们将一个同色垂直边的类型定义为一个鸽笼,其由一个确定的颜色和一个确定的行对组成。
    \begin{itemize}
        \item 颜色的可能性:$m$ 种。
        \item 行对的可能性(从 $m+1$ 行中选 2 行):$\binom{m+1}{2}$ 种。
    \end{itemize}
    鸽笼总数 $K$ 为:
    $$ K = m \binom{m+1}{2} $$
    我们将每一列视为一只鸽子。鸽子总数 $P$ 为:
    $$ P = C = m \binom{m+1}{2} + 1 $$

    \item 应用鸽巢原理:
    由于鸽子总数 $P$ 严格大于鸽笼总数 $K$ ($P = K + 1$),必有至少两只鸽子(设为列 $j_1$ 和 $j_2$,其中 $j_1 \ne j_2$)落入同一个鸽笼。
    这意味着存在一个颜色 $i^*$ 和一对行号 $(r_a, r_b)$,使得:
    \begin{itemize}
        \item 列 $j_1$ 中,方格 $(r_a, j_1)$ 和 $(r_b, j_1)$ 的颜色为 $i^*$。
        \item 列 $j_2$ 中,方格 $(r_a, j_2)$ 和 $(r_b, j_2)$ 的颜色为 $i^*$。
    \end{itemize}
    这四个方格
    $$ (r_a, j_1), \quad (r_b, j_1), \quad (r_a, j_2), \quad (r_b, j_2) $$
    构成了一个四角同为颜色 $i^*$ 的矩形。
\end{enumerate}
命题得证。

\question 设有正整数列 $\{a_1, a_2, \ldots, a_{77}\}$,其中任意连续 $7$ 项之和不大于 $12$,证明数列中存在连续若干项之和为 $22$。

证明:记 $S_i = \sum_{j=1}^{i} a_j$,则 $S_i$ 是一个严格递增的正整数列,且有以下性质:
\[
S_0 = 0, \quad S_1 = a_1, \quad S_2 = a_1 + a_2, \quad \ldots, \quad S_{77} = a_1 + a_2 + \cdots + a_{77}
\]
由于每连续 7 项之和不大于 12,推得 $S_{i+7} - S_i \leq 12$。因此,$S_i$ 是一个严格递增的序列,并且有:
\[
S_0 \leq S_1 \leq S_2 \leq \cdots \leq S_{77} \leq 132
\]
(因为每7项之和不大于12,总和不大于 $12 \times 11 = 132$)。

接下来,定义两个集合:
\[
A = \{S_0, S_1, S_2, \ldots, S_{77}\}
\]
\[
B = \{S_0 + 22, S_1 + 22, S_2 + 22, \ldots, S_{77} + 22\}
\]
显然,$A$ 和 $B$ 都是有限集合。由于 $S_i$ 的最大值为 132,所以集合 $A$ 的元素是从 0 到 132 中的 78 个元素,因此:
\[
|A| = 78
\]
而集合 $B$ 中的每个元素都是 $S_i + 22$,即从 22 到 154 的 78 个元素,因此:
\[
|B| = 78
\]
因此,集合 $A \cup B$ 中共有:
\[
|A \cup B| = 78 + 78 = 156
\]
而集合 $\{0, 1, 2, \ldots, 154\}$ 总共有 155 个元素。

根据鸽巢原理,必然存在 $S_i \in A$ 和 $S_j + 22 \in B$,使得:
\[
S_i = S_j + 22
\]
即:
\[
S_i - S_j = 22
\]
因此,存在连续若干项之和为 22,即:
\[
a_{j+1} + a_{j+2} + \cdots + a_i = 22
\]

综上所述,命题得证。

\question 证明或证伪:序列 $\{23, 2323, 232323, \ldots\}$ 中存在一个数能被 $233$ 整除。

序列中的第 $n$ 个数 $a_n$ 可以表示为:
$$a_n = \sum_{i=0}^{n-1} 23 \times 100^{i} = 23 \times \frac{100^{n} - 1}{99}$$
取序列的前 234 项,根据鸽巢原理,必然存在 $1\leq j \leq i$,使得 $a_i$ 和 $a_j$ 模 233 的余数相同,即, $a_i-a_j$ 可以被 233 整除
整理 $a_i-a_j$,与序列中的另一个数 $a_{i-j}$ 相关:
$$a_i-a_j = 23 \times \frac{100^{i} - 100^{j}}{99} = 23 \times \frac{100^{j}(100^{i-j} - 1)}{99} = 100^{j} a_{i-j}$$
s由于 100 与 233 互质,因此 $100^{j}$ 不能被 $233$ 整除,因此 $a_{i-j}$ 被 $233$ 整除。
命题得证。

\question
有 $101$ 个正整数,其和为 $300$,证明其中某些数之和恰好为 $200$。

记这些正整数为 $a_i\ (1\leq i\leq 101)$,记 $S_i = \sum_{j=1}^{i} a_j$。

考虑 $S_i \mod 100$的余数,根据鸽巢原理,必然存在 $i \neq j\ (1\leq j<i\leq 101)$ 使得 $S_i$ 和 $S_j$ 模 100 的余数相同,即 $S_i - S_j$ 可以被 100 整除.

显然,$S_i - S_j < 300$,分情况讨论:
\begin{enumerate}
    \item $S_i - S_j = a_{j+1} + a_{j+2} + \cdots + a_i = 0$,与 $a_i$ 为正整数矛盾,舍去。
    \item $S_i - S_j = a_{j+1} + a_{j+2} + \cdots + a_i = 100$,则 $a_1 + a_2 + \cdots + a_j + a_{i+1} + \cdots + a_{101} = 200$。
    \item $S_i - S_j = 200$,则 $a_{j+1} + a_{j+2} + \cdots + a_i = 200$。
\end{enumerate}
综上,命题得证。

\question 设 $S = \{1, 2, \ldots, 10^6\}$, $A \subseteq S$, $|A| = 101$。证明:总能找到 $B \subseteq S$,满足 $|B| = 100$,且集合 $\{a + b \mid a \in A, b \in B\}$ 中包含恰好 $|A| \cdot |B| = 10100$ 个元素。


\textbf{证明:}
记 $A = \{a_1, a_2, \ldots, a_{101}\}$,$B = \{b_1, b_2, \ldots, b_{100}\}$,逐步确定 $b_i$ 的取值,保证任意 $a_i+b_j$ 不重复。

\begin{enumerate}
    \item $b_1$ 可以取任意值。
    \item $b_2$ 需要满足 $ \forall i \neq j, \ a_i + b_1 \neq a_j + b_2$,即 $b_2 \neq b_1 + (a_i - a_j)$。 假设存在 $b_2 = b_1 + (a_i - a_j)$,此时 $b_2$ 有 $101 \times 100 = 10100$ 种取值;再排除 $a_i=a_j$ 即 $b_2 = b_1$,所以 $b_2$ 有 $10^6 - 10101$ 种取值。
    \item $b_3$ 需要满足 $ \forall i \neq j, \ a_i + b_1 \neq a_j + b_3$ 且 $\forall i \neq j, \ a_i + b_2 \neq a_j + b_3$,即 $b_3 \neq b_1 + (a_i - a_j)$ 且 $b_3 \neq b_2 + (a_i - a_j)$。
    同理,若存在 $i, j$ 使得 $b_3 = b_1 + (a_i - a_j)$ 或 $b_3 = b_2 + (a_i - a_j)$,则 $b_3$ 有 $2 \times 101 \times 100 = 20200$ 种取值。再排除 $b_3 = b_1$ 或 $b_3 = b_2$,则 $b_3$ 有 $10^6 - 20202$ 种取值。
    \item 以此类推,$b_k$ 需要满足 $\forall t < k, \forall i \neq j, \ a_i + b_t \neq a_j + b_k$,即 $b_k \neq b_t + (a_i - a_j)$。最后,对于 $b_{100}$,满足 $\forall t < 100, \forall i \neq j, \ a_i + b_t \neq a_j + b_{100}$,即 $b_{100} \neq b_t + (a_i - a_j)$。$b_{100}$ 有 $99 \times 101 \times 100 = 999900$ 种取值,再排除 $b_{100} = b_t$ 的情况,则 $b_{100}$ 有 $10^6 - 999999 = 1$ 种取值。
\end{enumerate}
综上,命题得证。

\question 设 $S$ 是正整数集合 $\{1, 2, \ldots, 2025\}$ 的一个子集,满足:对任意 $a, b \in S$(允许 $a = b$), $a + b$ 的十进制表示中不出现数字 $9$。求 $|S|$ 的最大值,并证明。

\textbf{证明:}
记$T = \{1, 2, \ldots, 2025\}$,构造 $S_0 \subseteq T$ 如下:
\[S_0 = \{x \in T \mid \text{所有数字}x_k \in \{0,1,2,3,4\}\}\]
$S_0$ 中任意 $a + b$ 的十进制表示中不出现数字 $9$。

计算 $|S_0|$:
\begin{enumerate}
    \item 1 位数:集合 $\{1, 2, 3, 4\}$,共有 $4$ 个。
    \item 2 位数:形如 $d_1 d_0$,其中 $d_1 \in \{1, 2, 3, 4\}$,$d_0 \in \{0, 1, 2, 3, 4\}$,共有 $4 \times 5 = 20$ 个。
    \item 3 位数:形如 $d_2 d_1 d_0$,其中 $d_2 \in \{1, 2, 3, 4\}$,$d_1, d_0 \in \{0, 1, 2, 3, 4\}$,共有 $4 \times 5 \times 5 = 100$ 个。
    \item 4 位数:形如 $d_3 d_2 d_1 d_0$,其中:
    \begin{itemize}
        \item 若 $d_3 = 1$,则 $d_2, d_1, d_0 \in \{0, 1, 2, 3, 4\}$,共有 $5 \times 5 \times 5 = 125$ 个。
        \item 若 $d_3 = 2$,则 $d_2 = 0$,$d_1 \in \{0, 1, 2\}$,$d_0 \in \{0, 1, 2, 3, 4\}$,共有 $1 \times 3 \times 5 = 15$ 个。
    \end{itemize}
\end{enumerate}
因此,$|S_0| = 4 + 20 + 100 + 125 + 15 = 264$,满足题意。

因此,$S$ 中元素个数的下界为 $264$。接下来证明 $|S|$ 的上界也是 $264$。

定义一个集合族 $\mathcal{P}$,其中每个集合 $P_{d_3 d_2 d_1 d_0}$ 满足以下条件:
\[
P_{d_3 d_2 d_1 d_0} = \{ e_3 e_2 e_1 e_0 \mid e_k = d_k \text{ 或 } e_k = 9 - d_k, \, k \in \{0, 1, 2, 3\} \}.
\]
即,集合 $P_{d_3 d_2 d_1 d_0}$ 包含了所有与 $d_3 d_2 d_1 d_0$ 在每个位上要么相等,要么相差 $9$ 的四位数。

进一步,我们定义集合族 $\mathcal{P}$ 为:
\[
\mathcal{P} = \{ P_{d_3 d_2 d_1 d_0} \mid d_k \in \{0, 1, 2, 3, 4\}, k \in \{0, 1, 2, 3\} \}.
\]
显然,对于集合 $P_{d_3 d_2 d_1 d_0}$ 中的任意两个元素 $a$ 和 $b$,$a + b$ 的十进制表示中必定包含数字 $9$。因此,在集合 $S$ 中,每个 $P_{d_3 d_2 d_1 d_0}$ 中最多只能包含一个元素,即每个集合中最小的元素 $d_3 d_2 d_1 d_0$,且满足 $d_3 d_2 d_1 d_0 \leq 2025$。(1)

此外,显然,$\mathcal{P}$ 中的各个集合是互不相交的,且它们的并集覆盖了所有从 $0$ 到 $9999$ 的四位数,即:
\[
\bigcup_{P \in \mathcal{P}} P = \{0, 1, 2, \ldots, 9999\}.
\]
因此,$S$ 中的元素一定是 $\mathcal{P}$ 中每个集合的最小元素,且这些最小元素的值不超过 $2025$。这就意味着,$S$ 中的元素实际上就是 $S_0$ 中的元素。因此,$|S| \leq |S_0| = 264$。

综上所述,$|S|$ 的最大值为 $264$。
